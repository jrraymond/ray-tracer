\documentclass[12pt]{article}

\usepackage{fancyhdr, amssymb, amsthm, amsmath, enumerate, marginnote}
\pagestyle{fancy}

% \fancyhf{}
\lhead{COMP356}
\chead{Final Project Proposal: Advanced Ray Tracing}
\rhead{Brian Gapinski // Justin Raymond}

\setlength{\headheight}{15pt}

\begin{document}
We will implement a ray-tracer with enhancements that provide user interactivity and photorealistic rendering.

The basic ray-tracer we wrote in homework 2 had the following features:
\begin{itemize}
    \item Anti-Aliasing

    \item Soft-Shadows

    \item Reflections
\end{itemize}

We propose to add the following enhancements:
\begin{itemize}
    \item Transparent and refractive materials

    \item Depth of Field

    \item Glossiness

    \item Motion Blur

    \item Being written in Haskell
\end{itemize}

\end{document}
